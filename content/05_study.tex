\chapter{Analysis and Results}
\label{chapter5}

\section{Analysis Strategy}

Throughout this analysis, steps are taken to determine the signal strength $\mu_{\rm{s-chan.}}=\sfrac{\sigma_{\rm{s-chan.}}^{\rm{measured}}}{\sigma_{\rm{s-chan.}}^{\rm{theory}}}$ of the single top quark $s$-channel process. 
This is done via a binned likelihood fit, where a \textit{Deep Neural Network} (DNN) output is used as discriminant value, to descern signal events from background events.
Only one signal region, pulled from the event selection in \autoref{tab:eventsel}, is defined in this thesis.
As systematic uncertainties are the limiting factors in the search for the single top quark $s$-channel process, the systematics with the biggest impact are further inspected and discussed. 


\section{DNN Model}

Especially in studying the single top quark $s$-channel process, a modern machine learning approach is sensible as the small signal to background ratio ($S/B 3.4 \%$) \cite{Seema:2706904} makes traditional methods difficult. 
Therefore a DNN output will be used as the discriminant for a binned likelihood fit.
The applied DNN was designed by Niklas Düser as part of his Master's Thesis at the \textit{Technische Universität Dortmund} \cite{niklas_master} and makes use of several high-level kinematic variables to descern signal events from background events. 
For every single particle that gets reconstructed, the kinematic variables $\pT, \eta, \varphi$ and the invariant mass $m$ get included as features for the DNN model.
Whereas for particle pairs their pseudorapidity difference $\Delta \eta$ and the difference in azimuthal angle $\Delta \varphi$ between them is incorporated as feature.
From the detector measurements of the final state particles, every other particle can be retraced up to the virtal $W$-boson. 
First, the final state lepton and corresponding neutrino are reproduced. 
The neutrino cannot be measured directly via the detector but is inferred via studying the missing transverse momentum $\etmiss$ of the event and further implying the longitudinal impulse of the neutrino, from the condition that the leptons and the neutrinos momentum squared must equal the invariant mass of the decaying $W_{\rm{Decay}}$-boson.
With the lepton and neutrino reconstructed, the $W_{\rm{Decay}}$-boson of the top quark decay can be solved for.
The leading b-tagged jet, of the top quark decay gets measured directly by the detector.
With information about the leading b-jet and the $W_{\rm{Decay}}$-boson, the top quark can be reconstructed.
From detector measurements, the subleading b-jet from the s-channel process itself gets assessed, and with the reconstructed top quark, the last possible step is to reconstruct the virtual $W_{\rm{Prod.}}$-boson.
Further the transverse mass of the $W_{\rm{Decay}}$-boson defined as $\mtw = \sqrt{2 \pT^l \etmiss (1-\cos \Delta\varphi)}$, with $\pT^l$ the transverse momentum of the lepton decay product and $\Delta\varphi$ the angle between lepton and neutrino, is included as feature.
As well as the transverse mass of the top quark, defined as $m_{\rm{T}}^{\rm{top}} = \sqrt{E_{\rm{T,sum.}}^2 - p_{\rm{T,sum.}}^2}$, with the sums of the transverse energies and transverse momenta of the top quarks decay particles, being the lepton, neutrino and $b$-quark.
The scalar sum of the transverse energies of all final-state paricles is called $H_{\rm{T}}$, whereas $H_{\rm{T,had.}}$ and $H_{\rm{T,lep}}$ sum only over the transverse energies of the final state hadrons or leptons respectively.
Another set of event shape variables are inferred through the momentum tensor, that encodes the high-levels momentum distributions of the reconstructed objects. Its elements are derived by the formular

\begin{equation*}
    M_{ij}=\frac{\sum^N_{k=1} p_{k,i}p_{k,j}}{\sum^N_{k=1} p_k^2}, ~ i,j \in x,y,z
\end{equation*}

by summing over all final-state particles, where $p_{k,i}$ denotes the $i$-th component of the $k$-th final state particle. 
Through the sorted eigenvalues of the momentum tensor $\lambda_0 > \lambda_1 > \lambda_2$ further event variables are constructed:

\begin{align*}
    &\text{Aplanarity}:  & \mathcal{A} &= \frac{3 \lambda_2}{2}, \\
    &\text{Planarity}:   & \mathcal{P} &= \frac{3(\lambda_2 + \lambda_1)}{2}, \\
    &\text{Spericity}:   & \mathcal{S} &= \lambda_1 - \lambda_2. \\
\end{align*}


The model was trained on both fullsim and fastsim datasets, as well as including single top quark $s$-channel samples with varying top quark masses for increased statistics.
Angular variables are cosine transformed to avoid discontinuities at $\pm \pi$ and variables carrying long tails are log-transformed to compress their dynamic range.
All variables are transformed into standardised units, to avoid distortion of variable importance due to their absolute values.
Data is split into even and odd samples, only in respect to their event number without taking any other variable feature into consideration. 
The DNN model outputs a value between 0 and 1, where 1 marks an event as totally corresponding to the signal and 0 marks the event as background. 
With the neural network output $x$ as the discriminant, the expected number of events $\lambda_i (\mu, \theta) = \mu \cdot s_i (\theta) + b_i (\theta)$ is used to incorporate the signal strength. 
Here $s_i(\theta)$ and $b_i(\theta)$ signify the expected signal and background contributions for the i-th bin, which are gathered by putting signal and background events through the DNN model and evaluating the output score $x$.
Systematic uncertainties can influence the seperation. 
This is considered through the nuisance parameters (NP) $\theta$, which will be affected further through fitting the binned likelihood. 
The signal strength $\mu$ serves then as a scaling factor for the expected signal yields across all bins. 
The DNNs output is shown in the form of the seperation plot in \autoref{subfig:seperation} and the combined DNN output histograms of each respective process shown in \autoref{subfig:histos}. The event yields given out from the DNN are shown in \autoref{tab:yields}.

\begin{figure}
    \centering
    \begin{subfigure}{0.48\textwidth}
        \centering
        \includegraphics[width=\textwidth]{imgs/seperation.png}
        \caption{The seperation plot shows the output score of the DNN on the $x$-axis for the $s$-channel signal histogram and the sum of the background histograms. Both histograms are normalised. A seperation of $30.1 \%$ is achieved, displayed by the observably seperated peaks with yet significant overlap of both distributions.}
        \label{subfig:seperation}
    \end{subfigure}
    \hfill
    \begin{subfigure}{0.48\textwidth}
        \centering
        \includegraphics[width=\textwidth]{imgs/SR-Schannel.png}
        \caption{The combined DNN output for every process of interest in this analysis. Histograms are stacked, where the signal yield is displayed in \textcolor{red}{red} on the top of the stack.}
        \label{subfig:histos}
    \end{subfigure}
    \caption{The DNNs output is shown with a seperation plot and the combined DNN output for the different processes }
    \label{fig:dnn_output}
\end{figure}


\section{Statistical Method}

To determine the signal strength, a binned profile likelihood fit is performed with the TRExFitter interface for the HistFactory tool (citation), which modifies the signal strength and the NPs to maximize the likelihood fuction. 
The likelihood function states the probability by which an outcome happens in respect of a set of given model parameters. 
It is used as a framework to determine the signal strength parameter $\mu$ and the various NPs of the different systematic uncertainties for which the highest probability is achieved, where the single top quark $s$-channel process is most likely to occur.
As High Energy Particle Physics (HEP) generates large amounts of data, histograms where data is gathered into bins, instead of encompassing each data point individually, are of use. 
The binning of data also introduces Poisson probabilities into the likelihood as can be seen in the following equation.

\begin{equation*}
    \mathcal{L}(\mu, \theta) = \prod_{i=1}^{N_{\rm{bins}}} \rm{Pois}(n_i|\lambda_i(\mu, \theta)) \cdot \prod_{j=1}^{N_{\rm{NP}}}\pi_j (\theta_j). 
\end{equation*}

The binned likelihood function multiplies the Poisson probabilities $\rm{Pois} (n|\lambda)$ in each bin, which state the probability of observing $n$ events under the expectation $\lambda$

\begin{equation*}
    \rm{Pois} (n|\lambda) = \frac{\lambda^n e^{-\lambda}}{n!},
\end{equation*}

multiplied with the product of all constraint terms $\pi_j (\theta_j)$ for each NP $j$.
A NP of $\theta_j = 0$ conforms to the nominal prediction made by the simulation, while $\theta_j = \pm 1$ varries the NP in one standard deviation $\pm \sigma$.
The constraint term can be modelled as a Gaussian function, where $\theta_j^0$ is set to the nominal value and  $\sigma_{\theta_j^0}$ to the uncertainty estimate. 
One standard deviation estimates for the uncertainties are provided as histograms. 
NPs are also included as free parameters, called as normalization factors (NF).
These are handled as multiplicative factors for signal or background components in the shape of $b(\vec{\theta},k) = k \cdot b (\vec{\theta})$, which is done for the $W$+jets process.
Statistical uncertainties stemming from the limited number of simulated events are included as NPs $\gamma_i$ for each respective bin $i$ with a nominal value $1$ and modelled as a Poisson term.    
By maximizing the likelihood function via variaton of the parameters $\mu$ and $\theta_j$, their final values are determined by $\displaystyle \hat{\mu}, \hat{\theta} = \operatorname*{argmax}_{\mu, \theta} \mathcal{L}(\mu, \theta)$.
Typically the negative logarithm of the likelihood function $-\ln \mathcal{L}$ is minimized, as the maximum does not move when taking the logarithm and minimization is often computationally more efficient.
The negative log-likelihood (NLL) function is fitted with \textsc{Minuit} (citation).
$\mu$ and all NPs can vary while fitting, where $\mu$ and the NFs are free-floating, while the NPs are penalized based on their respective uncertainties.
If after the fit the NP is changed from the nominal value, the data has pulled the NP. 
If after the fit the uncertainty of a NP is smaller then the provided uncertainty $\sigma_{\theta_j^0}$, the data achieved a constrainment on the systematic uncertainty provided by the NP. 
Thus the binned profile likelihood forms a framework for implementing the DNN output as discriminant and enabling the determination of the signal strength $\mu$ and the systematics encoded inside the NPs $\theta_j$ 

\section{Systematic Uncertainties}

A rough categorisation of the systematic uncertainties yield single top $s$-channel signal modelling uncertainties, background modelling uncertainties, where the $\ttbar$ process is especially studied, and detector modelling uncertainties. 

Modelling uncertainties occur because of different choices for ME matching and PS simulation by utilizing different generators and choosing values for special variables, that instruct these generators. 
For the single top quark $s$-channel signal process, additional samples were produced varying the $\pT^{\rm{hard}}$ to 1, changing ME emissions to be specified by the transverse momentum given by \textsc{Powheg} when utilized by \textsc{Pythia}.
This systematic then tests the ME matching descriptions of the hardest emission and further the subsequent PS. 
Additionaly a sample with \textsc{Herwig} was produced, assesing uncertainties stemming from the different definition for emission hardness, and choice of parton shower and hadronisation model in the individual generators.
Another sample is produced with the same ME and PS models as the nominal samples, but in the fastsim regime to evaluate uncertainties from different detector modelling procedures. 
Similarly, to attain further precision the for the largest background, futher samples were generated for the $\ttbar$ process.
Again, instead of \textsc{Pythia}, \textsc{Powheg} is interfaced with \textsc{Herwig} to study ME matching systematics. 
Another set of samples with $\pT^{\rm{hard}}=1$ is produced for the $\ttbar$ process. Lastly samples where the $h_{\rm{damp}}$ parameter is increased to $3 \cdot m_{\rm{top}}$ from $1.5 \cdot m_{\rm{top}}$ are generated, which enables studying of uncertainties related to damping radiation with high $\pT$ in the hardest emissions.

Additionaly to the ME and PS matching systematic uncertainties, each background process carries uncertainties in their cross-section. 
For the single top quark production modes an uncertainty of $6\%$ is set.
The $W$+jets and $Z$+jets background processes are estimated to have an cross-section uncertainty of $50\%$.
As $\ttbar$ does have the biggest yield, it is sensible to let its cross-section uncertainty be modelled as a free floating normalisation factor.


The detector modelling uncertainties in this study encompass one-sided and two-sided systematics, where two-sides systematics entail an up and down variation for the positive and negative pull of each respective uncertainty.
One-sided systematic uncertainties entail two systematics for the missing transverse momentum $\etmiss$, which parametrise the Track Soft Term (TST) resolution in parallel and perpendicular components. 
Two-sided systematics are more numerous.
The luminosity uncertainty is given as $0.83\%$.
Jet uncertainties come in form of jet energy resolution (JER), jet vertex tagger (Jvt) and jet energy scale (JES) uncertainties. 
JER uncertainties containt effects of noise and campaign difference pre-recommendations, effects from the difference of data and simulation gathered from the MC20 campaign, and various NP modelling uncertainties.
For Jvt two different efficiency uncertainties are included. 
The JES is effected by jet flavour composition and response, $b$-jet energy scale, various NP modelling effects on a detector, modelling or mixed level. 
Further, the effect of pile-up, punch through and jet $\eta$ intercalibration on the JES are included, as well as different pre-recommendations for different further effects on the JES. 
Many uncertainties for the effect of flavour tagging are considered, especially for $b$- and $c$- tagging.
Besides jet uncertainties, effects of uncertainties on the electron and muon modelling are taken into account, as well as the effect of pile-up reweighting. 
Uncertainties for the single top quark $s$-channel signal and $\ttbar$ background modelling stem additionaly from renormalisation and factorisation effects, moreover initial and final state radiation modelling uncertainties are regarded. 


In order to achieve independence from statistical fluctuations, the systematics distributions are smoothed. 
For uncertainties containing an up and down variation, two-sided symmetrization is done. 
The up and down variation for each respective bin $i$ consists therefore of the mean of both given by $\pm \frac{n_i^{\rm{up}} - n_i^{\rm{down}}}{2}$.
Uncertainties with only an up or down variation undergo one-sided symmetrisation, in which the the existing variation is mirrored for the missing variation.  

\section{Results}

After fitting the NLL onto the Asimov Dataset (ich muss nochmal erklären was ein Asimov Datensatz ist), the signal strength results in $\mu_{\rm{s-chan.}} = 1^{+0.25}_{-0.20}$.
The DNNs output is therefore still consistent with the SM prediction, even after incorporating several more systematic uncertainties into this analysis, as where considered in Niklas' work.
From the fit the NPs influences on the signal strength are now accessible. 
The ranking plot sorts the NPs, with the most influential NP shown on the top of the plot, based on their influence on the signal strength. 
The pre-fit impact of a NP is determined by calculating the shift in the signal strength between the nominal fit and a fit where the respective NP is fixed to the value $\hat{\theta} \pm \Delta \theta$. 
Here $\Delta \theta = 1$, which resembles the pre-fit uncertainty of the NP parameter of one standard deviation $\sigma$ as used in the constraint term.
For the post-fit impact a nominal fit and a fit where the NP is fixed with the value $\hat{\theta} \pm \Delta \hat{\theta}$ is compared.
$\Delta \hat{\theta}$ describes the uncertainty of the NP after the fit, with $\hat{\theta}$ being the value for the respective NP maximizing the likelihood function. 
The ranking plot for this analysis is displayed in \autoref{fig:ranking}

\begin{figure}
    \centering
    \includegraphics[width=0.5\textwidth]{imgs/ranking_plot.png}
    \caption{The ranking plot sorts the NPs from top to bottom regarding their influence on the signal strength.}
    \label{fig:ranking}
\end{figure}

The ranking plot reveals that the dominant uncertainty originates from the PS and hadronisation modelling of the single top quark s-channel process.
This is expected, since the hadronic signature of the signal consists only of two $b$-tagged jets, whose correlation depend strongly on the modelling of hard emissions and subsequent showering.
Furthermore, the features with the greatest impact on the DNNs output are the leading and subleading jets transverse momenta, explaining the strong correspondance to the PS and hadronisation modelling uncertainty. 
On the detector level the jet energy resolution uncertainties, given by noise pre-recommendation uncertainty and the effective NP 1 uncertainty, have a significant effect on the jet reconstruction and moreover on the jets momenta. 
Following, is the theoretical prior for the $\ttbar$ cross-section uncertainty and the $\mu_{\rm{R}}$ uncertainty for the $\ttbar$ process.
Since $\ttbar$ is the major top quark production mode and the major background in this analysis, the background modelling uncertainties show a significant impact. 
The small signal to backround ratio of the single top-quark s-channel process and the large cross-section of the $\ttbar$ process explain the sensitivity on the signal strength.
Especially the renormalisation uncertainty described by $\mu_{\rm{R}}$ effects the QCD emissions from the $\ttbar$ background, corresponding to alteriations in the kinematic variables of the background signature.
In combination with the high sensitivity of the DNN to these high-level kinematics, this explains its large impact.
Ratio plots of these NPs effects in contrast to the nominal prediction are displayed in \autoref{fig:ratioplots}.

\begin{figure}
    \centering
    % Ratio Plot: s-channel PS/hadr
    \begin{subfigure}{0.48\textwidth}
        \centering
        \includegraphics[width=0.95\textwidth]{imgs/syst/ps_sig.png}
        \caption{PS and hadronisation modelling uncertainty for the signal process. \\}
        \label{subfig:pshadwignal}
    \end{subfigure}
    \hfill
    % Ratio Plot: JERUnc_Noise_PreRec
    \begin{subfigure}{0.48\textwidth}
        \centering
        \includegraphics[width=0.95\textwidth]{imgs/syst/jer_noise.png}
        \caption{Jet Energy Resolution Noise pre-recommendation uncertainty for the signal process.}
        \label{subfig:jernoiseprerec}
    \end{subfigure}
    % Ratio Plot: ttbar cros section
    \begin{subfigure}{0.48\textwidth}
        \centering
        \includegraphics[width=0.95\textwidth]{imgs/syst/xsec_ttbar.png}
        \caption{Prior cross-section uncertainty for the $\ttbar$ background process.}
        \label{subfig:ttbarxsec}
    \end{subfigure}
    \hfill
    % Ratio Plot: JER_EffectiveNP_1
    \begin{subfigure}{0.48\textwidth}
        \centering
        \includegraphics[width=0.95\textwidth]{imgs/syst/jer_effective.png}
        \caption{Jet Energy Resolution effective NP 1 uncertainty for the signal process.}
        \label{subfig:jereffnp1}
    \end{subfigure}
    % Ratio Plot: ttbar mu_R
    \begin{subfigure}{0.48\textwidth}
        \centering
        \includegraphics[width=0.95\textwidth]{imgs/syst/mur_ttbar.png}
        \caption{Uncertainty for the renormalisation factor $\mu_{\rm{R}}$ for the $\ttbar$ background process.}
        \label{subfig:ttbarmur}
    \end{subfigure}
    \hfill
    % Ratio Plot: s-channel ME matching
    \begin{subfigure}{0.48\textwidth}
        \centering
        \includegraphics[width=0.95\textwidth]{imgs/syst/me_match_sig.png}
        \caption{ME matching uncertainty for the signal process.}
        \label{subfig:mesignal}
    \end{subfigure}
    \caption{Ratio plots for the six uncertainties with the biggest impact on the signal strength in this analysis.}
    \label{fig:ratioplots}
\end{figure}

The impact of the overall uncertainties are diminished by the DNN model in comparison the the MEM method used in the previous measurement. 
This is displayed in \autoref{tab:cat_impagt_sigxsecoversm}, where the procentual impact of the systematics grouped by categories are displayed.
These were determined similarly to the single impacts on shown in the ranking plot. 
Here the whole set of NPs corresponding to a category are fixed while fitting and then compared to full fit. 


\begin{table}
    \centering
    \caption{Observed impact of the uncertainties on the measured signal strength of the single top quark s-channel process grouped into categories.}
    \label{tab:cat_impagt_sigxsecoversm}
    \begin{tabular}{l c c}
        \toprule
        Systematic & Previous measurement: $\frac{\Delta \mu}{\mu} [\%]$ & Current measurement  $\frac{\Delta \mu}{\mu} [\%]$ \\
        \midrule
        Signal Modelling                & $+ 18/- 8$  & $+ 6/-6$ \\
        Jet Energy Resolution           & $+ 18/- 12$ & $+ 11/-9$ \\
        Jet Vertex Tagger               &  --     & $+ 1/-1$ \\
        Jet Energy Scale                & $+ 18/- 13$ & $+ 5/-4$ \\
        Flavor Tagging                  & $+ 12/- 10$ & $+ 4/-4$ \\
        Luminosity                      & $+ 4/- 3$   & $+ 1/-1$ \\
        Pile-up Reweighting             & $+ 5/- 3$   & $+ 4/-3$ \\
        \bottomrule
    \end{tabular}
\end{table}