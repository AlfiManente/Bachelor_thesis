% 01 Einleitung:

Ich mache erstmal Stichpunkte so wie Andrea mir das an die Hand gegeben hat.
Hangel mich aber an der Struktur von Noahs Arbeit entlang ohne diese zu kopieren.
Was schreibt Noah in seiner Einleitung?
\begin{itemize}
    \item Zunächst ein kleiner Absatz zum Standard model
    \item Etwas zum Top Quark selber, in Hinsicht zur Motivation, wieso überhaupt die Arbeit wichtig ist
    \item Was genau möchte ich messen? Wie sieht die Signatur aus? 
    \item Wie sieht unsere Selektion aus? 
    \item Was für einen Fit mache ich? Hier kann ich dann die DNN als Diskriminante und die profile likelihood selber erwähnen
    \item Dann kommt eine Aufzählung zu dem was in der Arbeit alles besprochen wird
    \begin{itemize}
        \item Overview of SM
        \item Overview about the top quark
        \item previos measurements explained
        \item short description of the ATLAS detector
        \item Sample and program definition
        \item object definition
        \item event selection
        \item main work of mine
        \item where are the results shown?
        \item short summary and outlook at the end
    \end{itemize}
\end{itemize}

Wie hat den Niklas seine Introduction strukturiert?

\begin{itemize}
    \item Er fängt mit dem Top Quark an, und deren eigenschaften und bildet somit einen Ankerpunkt für die Motivation des Schreibens
    \item Top Quark Eigenschaften:
    \begin{itemize}
        \item Top Quark Masse bei 172,5 GeV
        \item strong coupling to higgs field (was genau bedeutet das und wieso kann das wichtig sein?)
        \item important probe of electroweak symmetry, aber wieso?
        \item new physics beyond the sm!!
        \item Production with highest quantity is ttbar production through strong interaction (annihilation or gluon gluon fusion)
        \item Aber auch single channel über electroweak möglich!
        \item s channel kurz erklären
        \item t channel und tW gemessen nur s channel nicht mit bis zu 5\sigma
        \item expected and observed significances of prior measurement
        \item 
    \end{itemize}
\end{itemize}

Wie möchte ich dann meine Introduction strukturieren?

\begin{itemize}
    \item Wie kann direkt die Motivaton bzw das Gewicht dieser Arbeit beschreiben?
    \item Anfangen möchte ich einführend mit dem SM, so wie Noah am besten, aber ich möchte das nicht zu lang halten
    \item wieso ist das Top quark so wichtig und welche Eigenschaften hat es eigentlich?
    \item Wieso könnte eine nähere Untersuchung interessant sein?
    \item predominant production mode is ttbar über strong interaction
    \item dann kommen die single top channel über electroweak interaction
    \item s kanal wurde bis jetzt nicht präzise gemessen, signigicances erwähnen
    \item Grund dafür sind die hohen systematischen Unsicherheiten, wir haben trotzdem genug statistik
    \item Wie sieht die Signatur aus die wir im semileptonischen Kanal sehen wollen? Auch nur ein satz, turbo kurz
    \item Was ist meine Herangehensweise? Profile Likelihood mit DNN als Diskriminante mit besonderer Konzentration auf Systematics
    \item was für Daten benutze ich?
    \item Möchte ich kurz die Signatur ansprechen? 
    \item Dann Struktur der These
\end{itemize}

%02 Theory

\begin{itemize}
    \item introductory sentence: describes elementary particles and their interactions
    \item based on relativistic quantum field theory
    \item four fundamental forces: SM doesnt describe gravity
    \item dann  großer Absatz mit allen Teilchen aus der SM: Also Aufteilung in Fermionen mit spin 1/2 und gauge bosons mit spin 1 und scalar boson higgs boson
    \item bosons as medium of interactions
    \item fermion generations
    \item what forces do the different fermions experience?
    \item every fermion has an antimatter derivative (opposite charge and handedness(spin))
    \item First generation makes up all the matter (protons, neutrons and electrons)
    \item dann  auf die Bosonen eingehen
\end{itemize}

\section{The Top Quark}

\begin{itemize}
    \item Kobayashi and Maskawa postulated top quark
    \item Discovery at Fermilabs by CDF and D0 at tevatron p-antip collider
    \item Heaviest particle + state mass (near electroweak symmetry breaking scale??)
    \item Decay width and lifetime
    \item top quark doesnt hadronise
    \item almost exclusive decay into W boson and b quark
    \item But spin information and kinematic porperties encoded in decay products
\end{itemize}

\subsection{s-channel Top quark production}

\begin{itemize}
    \item s-channel is our production mode of interest
    \item tt bar via gluon gluon fusion is predominiant mode
    \item s-channel has the smallest cross section
    \item Anti quarks from sea quarks in the pdf
    \item Explain PDFs to explain sea and partons
    \item How does the theoretical cross section get calculated? Via factorisation theorem
    \item s-channel feynman diagram
\end{itemize}


\subsection{Top quark decay}

\begin{itemize}
    \item decay via weak interaction
    \item decay into W boson and bottom quark because of CKM V_tb approx 1
    \item W boson can decay hadronically 2/3 or leptonically 1/3.  Hadronic channel has bigger phase space because of more color charge possibilities
    \item dileptonic, semileptonic or hadronic challenge
    \item hadronic decay can be detected through jets
    \item leptonic decay leads to undetected neutrino. This leads to E_t^miss(how is it calculated?)
    \item What do the different channel signatures look like?
    \item We use semileptonic channel because of multijet background
\end{itemize}

\subsection{Background Processes}

Hier kann ich im Prinzip alles aus der Präsi rein hauen

\begin{itemize}
    \item What is background (Processes with similar signature)
    \item What are the most relevant backgrounds
    \item Feynman diagramms of background
    \item erklären wann background als signature gesehen werden kann
    \item Also im Prinzip präsi. Bei Noah steht nicht viel 
\end{itemize}

\subsection{Previous measurement}


\begin{itemize}
    \item Diagramm der vorherigen  sinlge  top measurements rein
    \item Wichtigsten  Stichpunkte ur vorherigen messung
    \item measured cross section
    \item measured (expected) significance
    \item which data at which lumi
\end{itemize}

%03 Detector

Habe ich jetzt schon geschrieben, aber müsste auch so passen.

%04 Object def and event sel

%4.1 Monte Carlo sims

\begin{itemize}
    \item analysis by studying simulations of signal and background processes and seperating them (Als Intro zu dem Kapitel)
    \item Processes are simulated with different monte carlo simulators, which each simulate any process better or worse (Ist aber nicht nur in Bezug zu den Unsicherheiten, sondern auch so)
    \item what part is simulated with data? Multijet? (Stimmt, gucken wir uns eigentlich Multijet background an?)
    \item hard interaction with pertubative theory (Auch rein als Intro)
    \item strong coupling non-pertubative (hadronisation und asymptotic freedom sorgen dafür dass man das nicht mit ME simulieren kann )
    \item parton showers and hadronisation described by phenomological models
    \item PYTHIA: Lund String model, HERWIG: heavy cluster model,
    \item Matrix element calculation by POWHEG BOX, SHERPA or aMC@NLO
    \item EVTGEN for heavy flavour decays 
\end{itemize}

\begin{itemize}
    \item Welche Programme wurden für welche Samples von welchen Prozessen benutzt?
    \item Wo kann ich diese nachlesen? Die stehen im Container Namen und sind eigentlich eins zu eins bei Niklas schon beschrieben worden!!!
    \item Wieso wurden diese Programme und Generatoren für die verschiedenen Prozesse benutzt? (Kann ich mir im Prinzip auch von Niklas nehmen)
    \item Welche Prozesse habe ich jetzt genutzt?
    \begin{itemize}
        \item single top s-channel ist unser signal
        \item $\ttbar background$ dominant background
        \item single top $t$ channel
        \item single top $\overline{t}$
        \item $\overline{t}$ + W? Was ist das für ein Prozess? associated $tw$ production
        \item $t$ + W ? associated $tw$ production 
        \item W lep nu, mit veto. Wieso genau haben wir diese benutzt? W+jets
        \item z ee ? z + jets
        \item z mu mu?
        \item z tau tau?
    \end{itemize}
    \item was gibt es dort für wichtige Parameter? $p_\rm{T}^\text{hard}$? $h_\text{damp}$? (werden wir diese noch benutzen? Sind ja am simulieren)
    \item ATLAS Detector Response simulation vie GEANT4 (fullsim) or ATLfast(fastsim)
    \item Wie ist denn hier die Reihenfolge? Erstmal alle Programme und ihren nutzen erklären und dann Block artig welche samples womit simuliert worden sind. Dann muss ich die einzelnen Samples nicht begründen, da dies ja schon mit dem Intro abgefrühstückt ist!
\end{itemize}

%4.2 Object def

\begin{itemize}
    \item Nach welchen teilchen schauen wir in der Signatur? leptons (e,\mu), neutrinos through $\etmiss$, jets and b-tagged jets
    \item Ansprechen dass Tau nicht gemessen wird wegen short lifetime, aber decay particles can be possible
    \item Likelihood criteria? isolation criteria?
    \item $\pT$ und $|\eta|$ definitionen für tight and loose particles
    \item transverse and longitudinal impact parameter specification: $|z_0 \sin \theta|<\SI{0.5}{\mm}$ und $|d_0/\sigma(d_0)|<5$
    \item anti-$k_t$ jet algo?
\end{itemize}

%4.3 Event Selection

\begin{itemize}
    \item state preselection (TopCP)
    \item table with final selection (FF) 
    \item here only signal region, on longer term validation/contro region
    \item table with event yields
\end{itemize}

% 5 Analysis

% 5.1 Analysis Strategy

\begin{itemize}
    \item binned maximum likelihood für signal strength √
    \item only one signal region (Soll ich da noch groß drauf eingehen? Eigentlich reicht das doch in der event sel!)
    \item we discriminate signal from the background through DDN output written by Niklas Düser √
    \item ME method schwierig bei kleinem B/S √
    \item I will discuss the effects of different systematics on the signal extraction √
\end{itemize}

% 5.2 DNN Model

\begin{itemize}
    \item Da besonders s-channel sehr wenig signal hat (S/B sehr klein), sind moderne machine learning ansätze gut √
    \item DNN geschrieben von Niklas Düser für Masterarbeit an TU Dortmund? √
    \item Nutzt mehrere high level kinematic variables zur evaluation √
    \item Preprocessing ansprechen? (Habe ich in einem Satz) √
    \item two fold split ansprechen? √
    \item Model shape ansprechen? (sprengt das den Rahmen?)
    \item Fullsim and fastsim (more statistics) used for training (Habe ich auch in einem Satz erwähnt) √
    \item Seperation Power is so gesehen das was als model output ausgegeben wird (Nicht ganz, sonder ist eine Metrik welche die komplette Seperation bestimmt über alle bins)
    \item λ, also expected event yields nimmt dann model output um später signal strength μ fitten zu können √
\end{itemize}

% 5.3 Binned Likelihood Fit

\begin{itemize}
    \item Formel für profile binned likelihood aufstellen √
    \item poisson verteilung zeigen mit λ als erwartungswert √
    \item wir fitten μ durch λ und θ durch die constraint terms √
    \item μ^ und θ^ als die Werte, welche die Likelihood maximieren √
    \item negative log likelihood erwähnen √
    \item ratio test erwähnen?
    \item Hätte ich hier nochmal TREx Fitter erwähnen sollen?
    \item Hier mehr auf die Nuisance Parameter eingehen und wie diese als \pm σ skaliert werden.
    \item Welche NPs werden als Gaussian und welche als Poissonian gemoddeled?
    \item Welche Parameter sind jetzt free floating?
    \item Wenden wir einen Asimov Fit an bei den Systematic Plots?
\end{itemize}

% 5.4 Systematic Uncertainties

\begin{itemize}
    \item Wie kann ich meine Systematics grob kategorisieren? (Modelling systematics[PS,ME], background modelling(ttbar[PS,ME]), detector modelling eigentlich auch. Ist der Rest kleinkram?) √
    \item ME und PS Unsicherheiten durch die anderen Samples, das kann ich auch nochmal in Niklas these nachschauen (Wo denn? Chapter 9.2, auch 6.2) √
    \item Hier nicht die jeweiligen Samples aufzählen, sonder erklären welche variablen variiert werden bei den jeweiligen systematics √
    \item different cross section uncertainties. Welche haben wir vorher fest gesetzt und welche sind free floating?
    \item Welche Uncertainties haben wir sonst? Besonders die verschiedenen Jet uncertatainy kategrorien aufzählen
    \item smoothing und symmetrization ansprechen. 
    \item Systematic Tabelle hier schon einfügen oder in den results? (in den Results mit den jeweiligen pulls)
\end{itemize}

% 5.5 Results

\begin{itemize}
    \item 
\end{itemize}