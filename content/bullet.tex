% 01 Einleitung:

Ich mache erstmal Stichpunkte so wie Andrea mir das an die Hand gegeben hat.
Hangel mich aber an der Struktur von Noahs Arbeit entlang ohne diese zu kopieren.
Was schreibt Noah in seiner Einleitung?
\begin{itemize}
    \item Zunächst ein kleiner Absatz zum Standard model
    \item Etwas zum Top Quark selber, in Hinsicht zur Motivation, wieso überhaupt die Arbeit wichtig ist
    \item Was genau möchte ich messen? Wie sieht die Signatur aus? 
    \item Wie sieht unsere Selektion aus? 
    \item Was für einen Fit mache ich? Hier kann ich dann die DNN als Diskriminante und die profile likelihood selber erwähnen
    \item Dann kommt eine Aufzählung zu dem was in der Arbeit alles besprochen wird
    \begin{itemize}
        \item Overview of SM
        \item Overview about the top quark
        \item previos measurements explained
        \item short description of the ATLAS detector
        \item Sample and program definition
        \item object definition
        \item event selection
        \item main work of mine
        \item where are the results shown?
        \item short summary and outlook at the end
    \end{itemize}
\end{itemize}

Wie hat den Niklas seine Introduction strukturiert?

\begin{itemize}
    \item Er fängt mit dem Top Quark an, und deren eigenschaften und bildet somit einen Ankerpunkt für die Motivation des Schreibens
    \item Top Quark Eigenschaften:
    \begin{itemize}
        \item Top Quark Masse bei 172,5 GeV
        \item strong coupling to higgs field (was genau bedeutet das und wieso kann das wichtig sein?)
        \item important probe of electroweak symmetry, aber wieso?
        \item new physics beyond the sm!!
        \item Production with highest quantity is ttbar production through strong interaction (annihilation or gluon gluon fusion)
        \item Aber auch single channel über electroweak möglich!
        \item s channel kurz erklären
        \item t channel und tW gemessen nur s channel nicht mit bis zu 5\sigma
        \item expected and observed significances of prior measurement
        \item 
    \end{itemize}
\end{itemize}

Wie möchte ich dann meine Introduction strukturieren?

\begin{itemize}
    \item Wie kann direkt die Motivaton bzw das Gewicht dieser Arbeit beschreiben?
    \item Anfangen möchte ich einführend mit dem SM, so wie Noah am besten, aber ich möchte das nicht zu lang halten
    \item wieso ist das Top quark so wichtig und welche Eigenschaften hat es eigentlich?
    \item Wieso könnte eine nähere Untersuchung interessant sein?
    \item predominant production mode is ttbar über strong interaction
    \item dann kommen die single top channel über electroweak interaction
    \item s kanal wurde bis jetzt nicht präzise gemessen, signigicances erwähnen
    \item Grund dafür sind die hohen systematischen Unsicherheiten, wir haben trotzdem genug statistik
    \item Wie sieht die Signatur aus die wir im semileptonischen Kanal sehen wollen? Auch nur ein satz, turbo kurz
    \item Was ist meine Herangehensweise? Profile Likelihood mit DNN als Diskriminante mit besonderer Konzentration auf Systematics
    \item was für Daten benutze ich?
    \item Möchte ich kurz die Signatur ansprechen? 
    \item Dann Struktur der These
\end{itemize}

%02 Theory

\begin{itemize}
    \item introductory sentence: describes elementary particles and their interactions
    \item based on relativistic quantum field theory
    \item four fundamental forces: SM doesnt describe gravity
    \item dann  großer Absatz mit allen Teilchen aus der SM: Also Aufteilung in Fermionen mit spin 1/2 und gauge bosons mit spin 1 und scalar boson higgs boson
    \item bosons as medium of interactions
    \item fermion generations
    \item what forces do the different fermions experience?
    \item every fermion has an antimatter derivative (opposite charge and handedness(spin))
    \item First generation makes up all the matter (protons, neutrons and electrons)
    \item dann  auf die Bosonen eingehen
\end{itemize}

\section{The Top Quark}

\begin{itemize}
    \item Kobayashi and Maskawa postulated top quark
    \item Discovery at Fermilabs by CDF and D0 at tevatron p-antip collider
    \item Heaviest particle + state mass (near electroweak symmetry breaking scale??)
    \item Decay width and lifetime
    \item top quark doesnt hadronise
    \item almost exclusive decay into W boson and b quark
    \item But spin information and kinematic porperties encoded in decay products
\end{itemize}

\subsection{s-channel Top quark production}

\begin{itemize}
    \item s-channel is our production mode of interest
    \item tt bar via gluon gluon fusion is predominiant mode
    \item s-channel has the smallest cross section
    \item Anti quarks from sea quarks in the pdf
    \item Explain PDFs to explain sea and partons
    \item How does the theoretical cross section get calculated? Via factorisation theorem
    \item s-channel feynman diagram
\end{itemize}


\subsection{Top quark decay}

\begin{itemize}
    \item decay via weak interaction
    \item decay into W boson and bottom quark because of CKM V_tb approx 1
    \item W boson can decay hadronically 2/3 or leptonically 1/3.  Hadronic channel has bigger phase space because of more color charge possibilities
    \item dileptonic, semileptonic or hadronic challenge
    \item hadronic decay can be detected through jets
    \item leptonic decay leads to undetected neutrino. This leads to E_t^miss(how is it calculated?)
    \item What do the different channel signatures look like?
    \item We use semileptonic channel because of multijet background
\end{itemize}

\subsection{Background Processes}

Hier kann ich im Prinzip alles aus der Präsi rein hauen

\begin{itemize}
    \item What is background (Processes with similar signature)
    \item What are the most relevant backgrounds
    \item Feynman diagramms of background
    \item erklären wann background als signature gesehen werden kann
    \item Also im Prinzip präsi. Bei Noah steht nicht viel 
\end{itemize}

\subsection{Previous measurement}


\begin{itemize}
    \item Diagramm der vorherigen  sinlge  top measurements rein
    \item Wichtigsten  Stichpunkte ur vorherigen messung
    \item measured cross section
    \item measured (expected) significance
    \item which data at which lumi
\end{itemize}