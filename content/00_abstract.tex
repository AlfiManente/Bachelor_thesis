\thispagestyle{plain}

\section*{Kurzfassung}
In dieser Arbeit werden Studien zur Messung des Wirkungsquerschnitts für die Erzeugung einzelner Top-Quarks im s-Kanal 
bei Proton-Proton Kollisionen mit einer Schwerpunktsenergie von $\sqrt{s}=\SI{13}{\tera\electronvolt}$ durchgeführt.
Die Daten stammen aus Kollisionsexperimenten des LHC und wurden vom ATLAS Detektor in den Jahren 2015 bis 2018 aufgenommen. 
Zur Extrahierung des Wirkungsquerschnitts wird ein Profile Likelihood Fit verwendet, wobei als Diskriminante die Ergebnisse eines \textit{Deep Neural Networks}(DNN) fungieren. 
Die Signal Signatur besteht aus einem geladenen Lepton, entweder Elektron oder Myon, zwei b-tagged Jets und fehlendem transversalen Impuls.

Was mache ich denn eigentlich genau in meiner Arbeit?
% Ich müsste jetzt erstmal eine Ahnung haben, was genau ich jetzt eigentlich mache 
% in meiner Bachelorarbeit
Was könnte ich noch dazu schreiben? 
\begin{itemize}
    \item Run 2 Daten 
    \item integrierte Luminosität
    \item Wie funktioniert die DNN? Es nutzt high-level kinematic variables. Was ist das überhaupt?
    \item Mache deutlicher dass ich simulierte Daten verwende
    \item Erkenntnisse aus dem Resultat vielleicht hier noch mit rein schreiben?
    \item Welche Uncertainties behindern uns am meisten? Background modelling, signal modelling, detector modelling
    \item kleiner Wirkungsquerschnitt des s-kanals und großer hintergrund bei ähnlichen signaturen
    \item s kanal bis jetzt nicht präzise gemessen, andere single top channel schon präzise genug gemessen
    \item Machen wir einen Asimov Fit? Was ist das genau? 
\end{itemize}

\section*{Abstract}
\begin{foreignlanguage}{english}
The abstract is a short summary of the thesis in English, together with the German summary it has to fit on this page.
\end{foreignlanguage}
