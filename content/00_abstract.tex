\thispagestyle{plain}

\section*{Kurzfassung}

Diese Arbeit untersucht die Produktion einzelner Top-Quarks im s-Kanal bei einer Schwerpunktsenergie von $\sqrt{s}=\SI{13}{\TeV}$ unter Verwendung simulierter Proton--Proton--Kollisionsdaten des Run--2--Datensatzes des ATLAS-Detektors am Large Hadron Collider (LHC).
Die Produktion einzelner Top Quarks im s-Kanal ist durch einen kleinen Produktionswirkungsquerschnitt und große Untergrundbeiträge gekennzeichnet.
Zur Trennung von Signal- und Untergrundprozessen wird ein Deep Neural Network eingesetzt, dessen Ausgabe als Diskriminanzvariable in einem Profile-Likelihood-Fit verwendet wird.
Die Analyse konzentriert sich auf den leptonischen Zerfall des Top-Quarks mit einem Endzustand aus einem geladenen Lepton, fehlendem transversalem Impuls und zwei $b$-Quark-Jets.
Die Signalstärke wird mithilfe eines Asimov-Datensatzes zu
$\mu_{\mathrm{s\text{-}chan.}} = 1^{+0.25}_{-0.20}$
bestimmt und ist mit der Vorhersage des Standardmodells vereinbar.
Die dominierenden Unsicherheiten stammen aus der Modellierung der Partonenschauer und Hadronisierung, der Jet-Energieauflösung sowie der Modellierung des $\ttbar$-Untergrundes.


\section*{Abstract}
\begin{foreignlanguage}{english}

This thesis presents an analysis of single top-quark production in the s-channel at a centre-of-mass energy of $\sqrt{s}= \SI{13}{\TeV}$ using simulated proton--proton collision data corresponding to the Run~2 dataset recorded by the ATLAS detector at the Large Hadron Collider.
The single top quark s-channel production mode is characterised by a small production cross section and large backgrounds. 
%making its experimental measurement particularly challenging.
To enhance the separation between signal and background processes, a Deep Neural Network (DNN) classifier is employed and used as the discriminant in a binned profile likelihood fit.
The analysis focuses on the leptonic decay mode of the top quark, resulting in a final state consisting of one charged lepton, missing transverse momentum and two $b$ quark jets.
The signal strength of the single top-quark s-channel process is measured using an Asimov dataset and found to be
$\mu_{\mathrm{s\text{-}chan.}} = 1^{+0.25}_{-0.20}$, consistent with the Standard Model prediction.
The dominant sources of uncertainty arise from parton shower and hadronisation modelling, jet energy resolution effects, and the modelling of the $\ttbar$ background.



%Systematic uncertainties related to signal and background modelling, detector effects, and pile-up are incorporated as nuisance parameters in the statistical model.

%This study demonstrates the improved sensitivity achieved with modern machine learning techniques and outlines possible directions for further improvements, including the inclusion of additional systematic uncertainties and the use of control regions to better constrain background contributions.

\end{foreignlanguage}
