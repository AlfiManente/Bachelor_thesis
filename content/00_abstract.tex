\thispagestyle{plain}

\section*{Kurzfassung}
In dieser Arbeit werden Studien zur Messung des Wirkungsquerschnitts für die Erzeugung einzelner Top-Quarks im s-Kanal 
bei Proton-Proton Kollisionen mit einer Schwerpunktsenergie von $\sqrt{s}=\SI{13}{\tera\electronvolt}$ durchgeführt.
Die Daten stammen aus Kollisionsexperimenten des LHC und wurden vom ATLAS Detektor in den Jahren 2015 bis 2018 aufgenommen. 
Zur Extrahierung des Wirkungsquerschnitts wird ein Profile Likelihood Fit verwendet, wobei als Diskriminante die Ergebnisse eines \textit{Deep Neural Networks}(DNN) fungieren. 
Die Signal Signatur besteht aus einem geladenen Lepton, entweder Elektron oder Myon, zwei b-tagged Jets und fehlendem transversalen Impuls.


\section*{Abstract}
\begin{foreignlanguage}{english}
The abstract is a short summary of the thesis in English, together with the German summary it has to fit on this page.
\end{foreignlanguage}
