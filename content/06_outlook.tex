\chapter{Summary and Outlook}
\label{chapter6}

A study of the effects of the systematic uncertainties on the signal strength $\mu_{\rm{s-chan.}}$ of the single top quark s-channel process was done, as previous measurements were limited by them.
For this, simulations based on data of the Run2 dataset, taken between 2015 - 2018 at $\sqrt{s}=\SI{13}{\TeV}$ measured with the ATLAS detector, were used.  
Only the leptonic decay mode of the $W$-boson from the top quark decay was regarded, leading to a signal signature consisting of one lepton, its corresponding neutrino inferred through the measurement of missing transverse momentum $\etmiss$, and two b-tagged jets.
In this study a single signal region was regarded.
As statistical method a binned profile likelihood fit was performed on an Asimov dataset, where the output of a Deep Neural Network was used as discriminant variable.
The measured singal strength resulted in $\mu_{\rm{s-chan.}} = 1^{+0.25}_{-0.20}$.
The uncertainties with the biggest impact on the signal strength were further discussed.
These contained the PS and hadronisation modelling of the single top quark s-channel signal process, JER uncertainties on the detector level and background modelling uncertainties for the major background, being the $\ttbar$ process. 

In furhter analyses, 

\begin{itemize}
    \item keine control regions
    \item fake estimate
    \item anderer b-tagging working point
    \item welche systematics nochmal besser anschauen
\end{itemize}