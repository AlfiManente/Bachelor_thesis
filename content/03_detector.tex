\chapter{The ATLAS Detector}
\label{chapter3}

\begin{figure}
    \centering
    \includegraphics[width=0.5\textwidth]{imgs/ATLAS_schematic.png}
    \caption{Schematic overview of the ATLAS Detector and all its subsystems \cite{Bianchi:2837191}}
    \label{atlas_detector}
\end{figure}

As a general-purpose detector, ATLAS covers a solid angle range of $4\pi$.
Its cylindrical geometry, consisting of a central barrel and two endcaps, ensures uniform angular coverage.
The detector is composed of three main subsystems: the Inner Detector (ID), the Calorimeter System, and the Muon Spectrometer (MS).
Charged-particle trajectories are reconstructed in the ID, which is immersed in a $\SI{2}{\tesla}$ solenoidal magnetic field.
The curvature induced by the Lorentz force allows for momentum and charge measurements.
The calorimeter system comprises an electromagnetic calorimeter (ECAL) and a hadronic calorimeter (HCAL).
Electrons and photons trigger electromagnetic showers in the ECAL, enabling energy and position measurements, while hadrons are absorbed in the HCAL, where hadronic showers are used to determine jet energies and directions.
Muons at the LHC are minimally ionising particles and are therefor not absorbed by the ID and calorimeters.
From tracks in the ID and energy deposits in the MS, trajectory and energies are reconstructed, with additional help of toroidal magnets in the barrel and endcap.
A right-handed Cartesian coordinate system is used, with the $z$-axis along the beam pipe, the $x$-axis pointing towards the centre of the LHC ring, and the $y$-axis pointing upwards.
The azimuthal angle $\phi$ is defined in the transverse plane, while the polar angle $\theta$ is measured with respect to the beam axis.
The pseudorapidity differences $\Delta\eta = -\ln(\tan(\Delta\theta/2))$ are used due to Lorentz invariance.
Distances in the detector are defined as $\Delta R = \sqrt{(\Delta\eta)^2 + (\Delta\phi)^2}$, and the transverse momentum is given by $\pT = \sqrt{p_x^2 + p_y^2}$.
