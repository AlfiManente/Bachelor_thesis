\chapter{The ATLAS detector}
\label{chapter3}

As a general purpose detector, the ATLAS detector covers a 4$\pi$ range to detect all outgoing particles originating from the hard scattering of the $pp$ beams. 
The full angular coverage in the beams symmetry plane is granted through its design, with a central barrel and two endcaps.
The detector posseses a mass of 7000 tons in a length of $\SI{44}{\m}$ and diameter of $\SI{25}{\m}$.
Three main sub systems of the detector, are necessary to be able to detect the full range of outgoing particles. These consist of the inner detector (ID), the Electromagnetic calorimeter (ECAL) and the Hadronic Tile Calorimeter (HCAL).

\begin{itemize}
    \item 4 $\pi$ detector 
    \item central barrel and two endcaps
    \item Detector measurements
    \item magnet system
    \item inner detector 
    \item electromagnetic calorimeter
    \item hadronic calorimeter
    \item Muon spectrometer
    \item LHCs Coordinate System 
    \item pseudorapidity and distance between particles, definition of transverse momentum
\end{itemize}

Extra Punkte 

\begin{itemize}
    \item Infos zum LHC hinzufügen?
    \item Inner detector: Pixel Detector, Semiconductor detector, Transition Radiation Tracker
    \item Trigger system?
\end{itemize}

\begin{figure}
    \centering
    \includegraphics[width=0.9\textwidth]{imgs/ATLAS_schematic.png}
    \caption{Schematic overviel of the ATLAS Detector and all its subsystems \cite{Bianchi:2837191}}
    \label{atlas_detector}
\end{figure}