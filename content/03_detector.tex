\chapter{The ATLAS Detector}
\label{chapter3}

\begin{figure}
    \centering
    \includegraphics[width=0.6\textwidth]{imgs/ATLAS_schematic.png}
    \caption{Schematic overview of the ATLAS Detector and all its subsystems \cite{Bianchi:2837191}}
    \label{atlas_detector}
\end{figure}

As a general purpose detector, the ATLAS detector covers a solid angle range of 4 $\pi$.
% to detect all outgoing particles originating from the hard scattering interaction point. 
The full angular coverage in the beams symmetry plane is granted through its cylindrical design, with a central barrel and two endcaps.
%The detector posseses a mass of 7000 tons in a length of $\SI{44}{\m}$ and diameter of $\SI{25}{\m}$.
It consists of three main subsystems, which are the \textit{Inner Detector} (ID), the \textit{Calorimeter System} and the \textit{Muon Spectrometer} (MS). 
With the help of a $\SI{2}{\tesla}$ solenoid magnet, the trajectory of charged particles are bent using the Lorentz force and then reconstructed inside the ID.
%To measure both electromagnetic and hadronic signatures, 
The calorimeter system consists of an Electromagnetic Calorimeter (ECAL) and Hadronic Calorimeter (HCAL).
Inside the ECAL the incoming electrons and photons trigger showers, which ionize parts of the calorimeter and results in energy and direction measurements.
Similarly, the HCAL absorbs hadronic jets and triggers hadronic showers, leading to light emissions in the scintillator tubes of the HCAL, from which energy and direction of the jets are determined.   
%Through these light emissions the energy and direction of the hadron jets can be determined. 
As minimal ionizing particles,  muons pass all the previously mentioned subsystems and can only be detected in the MS. 
Here the toroidal magnets in the endcap and barrel parts of the detector bend their tracks and helps in trajectory and charge measurements.

%A spherical coordinate system aids in a Lorentz invariant translation of the particle trajectories, but for this the initial definition of the cartesian coordinate system is necessary.
Cartesian coordinates  within the LHC are defined as follows:
The $z$-axis is aligned with the beam pipe, the $x$-axis points to the center of the LHC and the $y$-axis points upwards to the beam plane.
To introduce spherical coordinates, $\phi$ is used to describe the angle between the $y$- and $x$-axis, while $\theta$ describes the angle between the $y$- and $z$-axis. 
Lorentz invariance can then be achieved by introducing the pseudorapidity $\eta = - \ln (\tan \frac{\theta}{2})$.
Particle distances within the detector are then defined by the value $\Delta R = \sqrt{\Delta\eta^2 + \Delta\phi^2}$.
Transverse momentum is calculated within the cartesian coordinates  with $\pT = \sqrt{p_x^2+p_y^2}$.


