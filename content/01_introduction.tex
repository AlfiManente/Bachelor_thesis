\chapter{Introduction}

The Standard Model (SM) of particle physics is the best descriptor for the behavior 
of elementary particles and the interactions between them.
Despite its  major  successes many phenomena remain unexplained, like the discovery of the Higgs boson \cite{higgs_disc1, higgs_disc2} or the top quark \cite{top_quark_disc1, top_quark_disc2}.
Physics beyond the Standard Model (BSM) is needed to understand the nature of 
dark matter, neutrino oscillations or the matter-antimatter disbalance. %\\
%
As the heaviest fermion with a mass of about \SI{172.5}{\GeV} \cite{topmass_estimate, topmass_prec}, the top quark has the strongest coupling to the Higgs field and may act as a potential probe for electroweak symmetry breaking. 
Therefore in it lies potential in uncovering BSM physics. 
Its short lifetime of \SI{5e-25}{\s} \cite{particle_rev} makes hadronization impossible, which does not enable a direct measurement of the top quark. 
It almost exclusively decays into a $W$-boson and a bottom-quark via the weak interaction. %\\
%
In proton-proton ($pp$) collisions at the Large Hadron Collider (LHC), top-quarks are predominantly 
produced in top-antitop quark pairs ($\ttbar$) via the strong interaction.  
However, top quarks can be produced singly via the electroweak interaction, with 
the single-top production in the s- and t-channel, as well as in $tW$ associated production. 
The s-channel has the smallest cross section of \SI{10.32}{\pico\barn} of all mentioned processes and is dominated heavily by background processes.
Unlike the other single top-quark production modes, the s-channel
has not yet been measured with a significance over $5 \sigma$ at the LHC, with the latest ATLAS measurement achieving an observed (expected) significance of 3.3(3.9). \cite{lastmeasure_atlas}.
It used a matrix element method (MEM) discriminant in a binned profile likelihood fit to data.
Even with its small cross section, the major hindrance in the measurement is the size of the systematic uncertainties. 
To get a more precise measurement, this thesis will use a more modern approach, with a DNN output as the discriminant for a profile likelihood fit. 
Simulated data from $pp$ collisions in the LHC at $\sqrt{s}=\SI{13}{\TeV}$ measured by the ATLAS detector is used, corresponding to the full Run 2 dataset with a luminosity of \SI{140}{\femto\barn^{-1}}. %\\
%
The structure of this work is as follows: 
Chapter \ref{chapter2} displays an overview of the SM, the s-channel production mode as well 
as the dominant backgrounds and the previous measurement. Chapter \ref{chapter3} presents the ATLAS detector.
Chapter \ref{chapter4} discusses the simulation, the object definitions and event selections. 
Chapter \ref{chapter5} describes the main study presented in this thesis.
Finally chapter \ref{chapter6} gives a short summary of the results and an outlook to further improvements in the measurement of the single top s-channel.   
 