\chapter{Introduction}

The Standard Model (SM) of particle physics is the best descriptor for the behaviour 
of elementary particles and the interactions between them.
Despite its  major  successes, like the discovery of the Higgs boson or the top quark, many phenomena remain unexplained. 
Physics beyond the Standard Model (BSM) is needed to make breakthroughs in 
dark matter, neutrino oscillations or the matter-antimatter disbalance. 

As the heaviest fermion with a mass of about \SI{172.5}{\GeV}, the top quark is the strongest coupler with the Higgs field and may act as a potential probe for electroweak 
symmetry breaking. Therefore in it lies potential in uncovering BSM physics. 
The challenge lies in its short lifetime of \SI{5e-25}{\s}, 
which makes hadronisation impossible and almost exclusively leads to a $W$-boson an 
a bottom-quark as decay products.   

At the ATLAS experiment of the Large Hadron Collider (LHC) top-quarks predominantly 
come in pairs of $\ttbar$ via the strong interaction.  
Although, top quarks can be produced singly via the electroweak interaction, with 
the single-top production in the s- and t-channel, as well as in $tW$ associated production. 
The s-channel has the smallest cross section of \SI{10.32}{\pico\barn} of all mentioned processes and is dominated heavily by background processes.
Unlike the other single modes, the s-channel
has not yet been measured precisely with a significance over $5 \sigma$, with 
the latest ATLAS measurement achieving an observed (expected) significance of 3.3(3.9).
Even with its small cross section, the major hindrance in the measurement is the systematic uncertainties. To get a better measurement, this thesis will use a more 
modern approach, with a DNN output as the discriminant for a profile likelihood fit. 
Simulated data from proton-proton collisions in the LHC at $\sqrt{s}=\SI{13}{\TeV}$ corresponding to the full Run 2 dataset with a luminosity of \SI{140}{\per\femto\barn} 
from the ATLAS detector is used. 

The structure of this work constitutes the following content. 
In \autoref{chapter2} an overview of the SM, the s-channel production mode as well 
as the dominant backgrounds and the previous measurement is given. \autoref{chapter3} presents the ATLAS detector.
\autoref{chapter4} lists the samples, the object definitions and event selections used. 
\autoref{chapter5} describes the main study presented in this thesis.
Finally \autoref{chapter6} gives a short summary of the results and an outlook for further advances in the measurement of the single top s-channel.   

%Each of these modes produces the top-quark in 
%a $tWb$ vertex, where the element $V_\rm{tb}$ of the Cabbibo-Kobayashi-Maskawa (CKM) matrix is involved.
 