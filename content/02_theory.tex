\chapter{Properties of the Top Quark within the Standard Model}
\label{chapter2}

\section{Overview of the Standard Model}
\label{overview_sm}

The SM of particle physics is a gauge invariant theory, which describes the fundamental particles and the interactions between them. 
It was developed throughout the 20th century by combining the discoveries of quantum mechanics and special relativity into a quantum field theory. 
Three of the four fundamental forces, electromagnetism, the strong interaction and the weak interaction are characterized by it, only excluding gravity. 
Whereas electromagnetism and gravity act on infinite ranges, the strong and weak force act on subatomic scales. 

The SM divides the fundamental particles into two main categories. 
First fermions, which constitute all known matter and carry half integer spin.  
Second bosons, which act out the interactions between fermions and carry a whole integer spin. 
Further, bosons can be divided into vector bosons, with spin 1 and a scalar boson, the Higgs boson ($H$) with spin 0. 
The fermions are comprised of quarks and leptons, where both interact with the electromagnetic (EM) force but only quarks experience the strong force. 
Neutrinos, which take part in the lepton category, only interact via the weak force. 
The fundamental particles are further split into three generations in relation to their masses and time of discovery. 
The first generation is comprised of the electron ($e$) and the corresponding electron neutrino ($\nu_\rm{e}$) for the lepton part, and the up quark ($u$) as well as the down quark ($d$). 
The second generation leptons include the muon ($\mu$) and the muon neutrino ($\nu_\rm{\mu}$), while the charm quark ($c$) and the strange quark ($s$) constitute the second generation quarks. 
The third generation includes the heaviest fermions, with the tau lepton ($\tau$) and its corresponding tau neutrino ($\nu_\rm{tau}$) on the lepton side and the top quark ($t$) and bottom quark ($b$) on the quark side. 
The electric charge defines the interaction strength with the EM force. All charged leptons posses an electric charge of -1 elementary charge (e). 
Whereas up-type quarks ($u, c, t$) carry a charge of $+\sfrac{2}{3} \, \rm{e}$ and down-type quarks ($d, s, b$) carry a charge of $-\sfrac{1}{3} \, \rm{e}$. 
For every fermion there exists an antiparticle counterpart with oppsite electric charge and spin. 
Especially the first generation of fermions constitute to the overwhelming part of matter, where protons and neutrons are build with the $u$- and $d$-quarks and electrons as part of atoms.

The force carrying bosons include the photon ($\gamma$) for the EM interaction, the $\rm{W}^{\pm}$ and $\rm{Z}$ bosons for the weak interaction and eight gluons with different colour for the strong interaction. 
The photon is mass- and chargeless, with it coupling to every particle possesing an electric charge. 
The $\rm{W}^{\pm}$  and $\rm{Z}$ bosons of the weak interaction carry relatively high mass, which is the reason for its narrow interaction range. 
The weak force acts on every fermion, where the $\rm{W}^{\pm}$ bosons couple to the weak isospin and the $\rm{Z}$ boson couples to the weak hypercharge, a combined value of the electric charge and weak isospin.   
Gluons, mediating the strong interaction, are massless as well and couple to any particle carrying a colour charge, either quarks or glouns themselves. 
There are the three base colours red, green and blue with a respective anticolour counterpart. Gluons always carry two colour charges, one base colour and one anticolour, resulting in eight different possible permutations. 
Quarks carry only one colour charge. 
Colour confinement states that every stable composite particle has to be colorless, which is the reason for the finite range of the strong interaction.
These stable composites are called hadrons and are further split into mesons and baryons. 
Mesons are colour neutral quark-anitquark pairs. Baryons are colour neutral composites of three quarks. 
The Higgs Boson, proposed by Peter Higgs in 1964 \cite{higgs}, was discovered in 2012 in a combined effort of the ATLAS \cite{atlas_higgs} and CMS \cite{cms_higgs} experiment at the LHC.
As a scalar boson it carries spin 0, no electric charge and is the heaviest boson ($m_\rm{H}=\SI{125.20 \pm 0.11}{\GeV}$ \cite{higgs_mass}). 
It mediates none of the fundamental forces, but by coupling to it fermions and the $\rm{W}^{\pm}$ and $\rm{Z}$ boson gain their mass. 


\begin{figure}
    \centering
    \includegraphics[width=0.7\textwidth]{imgs/SM1.png}
    \caption{The fundamental particles of the Standard Model, showing which forces act on them respectively \cite{UZH_SM}}
    \label{sm_table}
\end{figure}