\chapter{Properties of the Top Quark within the Standard Model}
\label{chapter2}

\section{Overview of the Standard Model}
\label{overview_sm}

The SM of particle physics is a gauge invariant theory, which describes the fundamental particles and the interactions between them. 
It was developed throughout the 20th century \, \cite{standard_model} by combining the discoveries of quantum mechanics and special relativity into a quantum field theory. 
Three of the four fundamental forces, electromagnetism, the strong interaction and the weak interaction are characterized by it, only excluding gravity. 
Whereas electromagnetism and gravity act on infinite ranges, the strong and weak force act on subatomic scales. 

The SM divides the fundamental particles into two main categories. 
First fermions, which constitute all known matter and carry half integer spin.  
Second bosons, which act out the interactions between fermions and carry a whole integer spin. 
Further, bosons can be divided into vector bosons, with spin 1 and a scalar boson, the Higgs boson ($H$) with spin 0. 
The fermions are comprised of quarks and leptons, where both interact with the electromagnetic (EM) force but only quarks experience the strong force. 
Neutrinos, which take part in the lepton category, only interact via the weak force. 
The fundamental particles are further split into three generations in relation to their masses and time of discovery. 
The first generation is comprised of the electron ($e$) and the corresponding electron neutrino ($\nu_\rm{e}$) for the lepton part, and the up quark ($u$) as well as the down quark ($d$). 
The second generation leptons include the muon ($\mu$) and the muon neutrino ($\nu_\rm{\mu}$), while the charm quark ($c$) and the strange quark ($s$) constitute the second generation quarks. 
The third generation includes the heaviest fermions, with the tau lepton ($\tau$) and its corresponding tau neutrino ($\nu_\rm{tau}$) on the lepton side and the top quark ($t$) and bottom quark ($b$) on the quark side. 
The electric charge defines the interaction strength with the EM force. All charged leptons posses an electric charge of -1 elementary charge (e). 
Whereas up-type quarks ($u, c, t$) carry a charge of $+\sfrac{2}{3} \, \rm{e}$ and down-type quarks ($d, s, b$) carry a charge of $-\sfrac{1}{3} \, \rm{e}$. 
For every fermion there exists an antiparticle counterpart with oppsite electric charge and spin. 
Especially the first generation of fermions constitute to the overwhelming part of matter, where protons and neutrons are build with the $u$- and $d$-quarks and electrons as part of atoms.

The force carrying bosons include the photon ($\gamma$) for the EM interaction, the $\rm{W}^{\pm}$ and $\rm{Z}$ bosons for the weak interaction and eight gluons with different colour for the strong interaction. 
The photon is mass- and chargeless, with it coupling to every particle possesing an electric charge. 
The $\rm{W}^{\pm}$  and $\rm{Z}$ bosons of the weak interaction carry relatively high mass, which is the reason for its narrow interaction range. 
The weak force acts on every fermion, where the $\rm{W}^{\pm}$ bosons couple to the weak isospin and the $\rm{Z}$ boson couples to the weak hypercharge, a combined value of the electric charge and weak isospin.   
Gluons, mediating the strong interaction, are massless as well and couple to any particle carrying a colour charge, either quarks or glouns themselves. 
There are the three base colours red, green and blue with a respective anticolour counterpart. Gluons always carry two colour charges, one base colour and one anticolour, resulting in eight different possible permutations. 
Quarks carry only one colour charge. 
Colour confinement states that every stable composite particle has to be colorless, which is the reason for the finite range of the strong interaction.
These stable composites are called hadrons and are further split into mesons and baryons. 
Mesons are colour neutral quark-anitquark pairs. Baryons are colour neutral composites of three quarks. 
The Higgs Boson, proposed by Peter Higgs in 1964\,\cite{higgs_theory}, was discovered in 2012 in a combined effort of the ATLAS\,\cite{higgs_disc1} and CMS\,\cite{higgs_disc2} experiment at the LHC.
As a scalar boson it carries spin 0, no electric charge and is the heaviest boson ($m_\rm{H}=\SI{125.20 \pm 0.11}{\GeV}$\,\cite{particle_rev}). 
It mediates none of the fundamental forces, but by coupling to it fermions and the $\rm{W}^{\pm}$ and $\rm{Z}$ boson gain their mass. 


\begin{figure}
    \centering
    \includegraphics[width=0.7\textwidth]{imgs/SM1.png}
    \caption{The fundamental particles of the Standard Model, showing which forces act on them respectively \cite{UZH_SM}}
    \label{sm_table}
\end{figure}

\section{The Top Quark}

First postulated by Makoto Kobayashi and Toshihide Maskawa in 1973 \cite{top_theory} and later discovered at Fermilab with the CDF and D\O \, experiments of the Tevatron collider in 1995 in $p\overline{p}$ collisions \cite{top_quark_disc1,top_quark_disc2}, the top quark is the last discovered quark in the SM and is the up type quark of the third generation.
It carries the heaviest mass of all quarks at $m_\rm{t}= \SI{172.95 \pm 0.53}{\GeV}$ \cite{topmass_prec} and posseses a large decay width of $\Gamma_{t}= \asym{1.42}{0.19}{0.15}{\GeV}$ \cite{topdecaywidth}. 
This large decay width leads to a short mean life time of $\tau = \sfrac{\hbar}{\Gamma_\rm{t}} \approx \SI{5e-25}{\s}$\cite{particle_rev}. 
Unlike other quarks, which hadronize due to colour confinement, the top quark decays before hadronization is possible, due to this short lifetime. 
The tops properties, like spin information or kinematic properties, are then to be decoded by studying the decay particles, which makes its analysis especially interesting. 
The top quark decays almost exclusively into a $W$-boson and a $b$-quark\,\cite{vtb_cdf}.   

\section{Single Top Quark Production in the s-Channel}

The s-channel is one of the electroweak top-production modes, producing a single top or antitop quark. 
In the single top-quark s-channel, a top-quark and a bottom anitquark are created from a virtual $W$-boson, after the annihilation of a quark and antiquark. 
The s-channel shows a $tWb$ vertex, with the CKM matrix element being almost one.
Therefore the s-channel is a strong probe for the electroweak properties of the top quark. 
At the LHC, the annihilating particles are predominantly valence up quarks ($u$) and antidown ($\overline{d}$) from the pool of sea quarks in the colliding protons.

(Unterschied zwischen $\overline{u}$ und $d$ als annihilierende Teilchen heraus finden)

The Feynman diagram of the s-channel process at LO in QCD is shown in \autoref{subfig:schan}.

(Den folgenden Teil erst bei der simulationen erwähnen)

The parton density function (PDF) is a representation of the probability distrubtion with wich a parton carries a fraction of the momentum of a hadron, denoted by using the \textit{Bjorken $x$}. 
$p_i = x p_p$ resembles the momentum fraction the parton $i$ carries from the whole proton momentum $p_p$. 
Where a parton is a hadron component, either a quark or gluon. 
The cross-section $\sigma_{\rm{s-chan.}}$ of the single top quark s-channel process, is determined using the factorisation theorem of the QCD.
By convolving the PDFs of the parton constituents $i$ and $j$ with the cross section $\sigma_{ij \rightarrow t\overline{b}}$ of the respective sub-process and then summing over all possible sub-processes, the s-channel cross section is determined. 

%occuring from the interaction of partons $i$ and $j$ , 
\begin{equation*}
    \sigma_{\rm{s-chan.}} = \sum_{i,j} \int \rm{d}x_i f_i (x_i, \mu_{\rm{F}}^2) \int \rm{d}x_j f_j (x_j, \mu_{\rm{F}}^2) \sigma_{ij \rightarrow t\overline{b}} (\sqrt{s}, m_t, \mu_{\rm{F}}, \mu_{\rm{R}}, \alpha_S) .  
\end{equation*}

The formular takes in, the centre-of-mass energy $\sqrt{s}$ of the process, the top quark mass $m_t$, the coupling strength of the strong interaction $\alpha_S$, as well as the factorisation scale $\mu_{\rm{F}}$ and the renormalisation scale $\mu_{\rm{R}}$, which are set to the energy scale of the process. 





% \begin{itemize}
%     \item faktorisierungstheorem als "aufhänger" (nachlesen)
%     \item s-channel is our production mode of interest
%     \item s-channel has the smallest cross section
%     \item Anti quarks from sea quarks in the pdf
%     \item Explain PDFs to explain sea and partons
%     \item How does the theoretical cross section get calculated? Via factorisation theorem
%     \item what does the signature look like?
% \end{itemize}


\begin{figure}
    \centering
    \begin{subfigure}{0.5\textwidth}
        \centering
        \begin{tikzpicture}
            \begin{feynman}
              \vertex (a);
              \vertex [above left=of a] (i1) {$u$};
              \vertex [below left=of a] (i2) {$\overline{d}$};
              \vertex [right=of a] (b);
              \vertex [above right=of b] (f1) {$t$};
              \vertex [below right=of b] (f2) {$\overline{b}$};
              % \vertex [above right=5mm and 1cm of f1] (c1) {$W^+$}; 
              % \vertex [below right=5mm and 1cm of f1] (c2) {$b$};
              % \vertex [above right=5mm and 15mm of c1] (d1) {$\overline{e}, \, \overline{\mu}$};  
              % \vertex [below right=5mm and 15mm of c1] (d2) {$\nu_e , \, \nu_{\mu}$};

              \diagram* {
              (i1) -- [ fermion] (a) -- [ fermion] (i2);
              (a) -- [boson, edge label={$W^+$}](b);
              (f2) -- [fermion] (b) -- [ fermion] (f1);
              % (c1) -- [boson] (f1) -- [fermion] (c2);
              % (d1) -- [fermion] (c1) -- [fermion] (d2);
              };

            \end{feynman}
        \end{tikzpicture}
        \caption{Single top-quark production s-channel}
        \label{subfig:schan}
    \end{subfigure}
    % \hfill
    \begin{subfigure}{0.5\textwidth}
        \centering
        \begin{tikzpicture}
            \begin{feynman}
              \vertex (a);
              \vertex [right=of a] (b);
              \vertex [above right=7mm and 15mm of b] (c1);
              \vertex [below right=of b] (c2) {$b$};
              \vertex [above right=8mm of c1] (d1) {$e^+, \, \mu^+$};
              \vertex [below right=8mm of c1] (d2) {$\nu_e, \, \nu_\mu$};

              \diagram* {
                (a) -- [edge label={$t$}, fermion](b);
                (c1) -- [edge label={$W^+$}, boson] (b) -- [fermion] (c2);
                (d1) -- [fermion] (c1) -- [fermion] (d2);
              };
            \end{feynman}
        \end{tikzpicture}
        \caption{Leptonic decay of the top quark}
        \label{subfig:topdecay}
    \end{subfigure}
    \caption{Feynman diagram of the single top quark s-channel production mode (left) and the leptonic decay of the top quark (right).}
    \label{fig:schan_feynman}
\end{figure}


\subsection{Top Quark Decay}

The top quark decay occurs via the weak interaction, mainly into a $b$-quark and a $W$-boson. 
Furthermore, the $W$-boson decays either hadronically $\sfrac{2}{3}$ or leptonically  $\sfrac{1}{3}$ of the times. 
The hadronic decay results into two quarks, whereas the leptonic decay results in a charged lepton and its corresponding neutrino.
The leptonic decay of the $W$-boson is shown in \autoref{subfig:topdecay}.
As there are different color permutations the quarks can carry, the phase space for the hadronic decay is larger, making it more likely than the leptonic decay.
During the s-channel analysis only leptonic $W$-boson decays are selected, as the final state from the hadronic decay suffers from multijet background production.
% Since neutrinos cannot be directly detected, the event has to be examined for missing transverse momentum $\etmiss$. 
If the sum of all transverse momenta of all the outgoing particles, does not add to zero, then a particle has remained undetected and the event has missing transverse momentum $\etmiss$, as prior to the collision.
Events with neutrinos in the final state are therefore expected to have missing transverse momentum. 
% both beam momenta should cancel out. The leptonic decay mode of the top quark, as looked for in this thesis, is shown in \autoref{subfig:topdecay}.

The searched for signature consist therefore of a charged lepton from the leptonic $W$-boson decay, missing transverse momentum $\etmiss$ from the undetectable neutrino, and two $b$ quarks, one produced from the s-channel process itself and one from the $b$-quark of the top-quark decay. 
%Hier nochmal erklären was jets sind. Später in der Object selection b-tagging erklären
$\tau$ leptons have a short lifetime and can thus not be directly detected, but first and second generation leptons from the $\tau$ decay can be observed. Because of that lepton signature is narrowed down to either an electron or muon.

\subsection{Background Processes}

There are physics processes with similar signatures in the detector as the signal process discussed above. 
Differentiating these processes from the process of interest is a major task in any analysis.
These so-called background processes consist predominantly of $\ttbar$ production and $W$+jets production for the single top s-channel analysis. 
In the LHC collisions, $\ttbar$ production occurs mostly through gluon-gluon fusion, but with less probability can also happen through quark-pair-annihilation.
With a theoretical cross-section of  $\sigma_{\ttbar}^{\rm{Theo.}}=\asym{832}{20}{29}{\pico\barn}$\,\cite{ttbar_theory_xsec}, it exceeds every other top quark production mode. 
The other single top quark production modes can be mistaken for signal as well, but to a lesser degree, since their cross sections are noticeably smaller than $\ttbar$ with $\sigma_{t-\rm{chan.}}^{\rm{Theo.}}=\asym{216.99}{9.04}{7.71}{\pico\barn} $ \cite{tchan_xsec} and $\sigma_{tW-\rm{Prod.}}^{\rm{Theo.}}=\SI{71.7 \pm 3.8}{\pico\barn}$ \cite{tw_xsec}.
Other minor backgrounds stem from multijet background processes, $Z$+jets or diboson($WW$,$WZ$,$ZZ$) production.
\autoref{fig:background_feyn} shows the LO Feynman diagrams for $\ttbar$ production via gluon-gluon fusion, and the other two single top-quark production modes.

\begin{figure}
    \centering
    \begin{subfigure}{0.3\textwidth}
        \centering
        \begin{tikzpicture}
            \begin{feynman}
                \vertex (a);
                \vertex [right=of a](b);
                % initial state
                \vertex [above left=of a] (i1) {$g$};
                \vertex [below left=of a] (i2) {$g$};
                % final state
                \vertex [above right=of b] (f1) {$t$};
                \vertex [below right=of b] (f2) {$\overline{t}$};


                \diagram* {
                (i1) -- [gluon] (a) -- [gluon] (i2);
                (a) -- [gluon, edge label={$g$}] (b);
                (f1) -- [anti fermion] (b) -- [anti fermion] (f2);
                };
            \end{feynman}
        \end{tikzpicture}
        \caption{$\ttbar$ production through gluon gluon fusion.}
    \end{subfigure}
    \hfill
    \begin{subfigure}{0.3\textwidth}
        \centering
        \begin{tikzpicture}
            \begin{feynman}
                \vertex (a);
                \vertex [below=of a] (b);
                \vertex [below left=3mm and 12mm of b] (c1) {$b$};
                \vertex [below right=3mm and 12mm of b] (c2) {$t$};
                \vertex [above left=3mm and 12mm of a]  (d1) {$u$};
                \vertex [above right=3mm and 12mm of a] (d2) {$d$};

                \diagram* {
                (d1) -- [fermion] (a) -- [fermion] (d2);
                (a) -- [boson, edge label=\(W^+\)] (b);
                (c1) -- [fermion] (b) -- [fermion] (c2);
                };
            \end{feynman}
        \end{tikzpicture}
        \caption{Single top t-channel. \\ ~ \\  }
    \end{subfigure}
    \hfill
    \begin{subfigure}{0.3\textwidth}
        \centering
        \begin{tikzpicture}
            \begin{feynman}
                \vertex (a);
                \vertex [above left=of a] (i1) {$b$};
                \vertex [below left=of a] (i2) {$g$};
                \vertex [right=of a] (b);
                \vertex [above right=of b] (f1) {$W^-$};
                \vertex [below right=of b] (f2) {$t$};
                    
                \diagram* {
                (i1) -- [fermion] (a) -- [gluon, opacity=0.6] (i2);
                (a) -- [fermion, edge label={$b$}](b);
                (f2) -- [anti fermion] (b) -- [boson, opacity=0.6] (f1);
                };
            \end{feynman}
        \end{tikzpicture}
        \caption{Associated $tW$ production.}
    \end{subfigure}
    \caption{Feynman diagrams at LO for the major background processes in the analysis of the single top quark s-channel production mode.}
    \label{fig:background_feyn}
\end{figure}

Not only does the $\ttbar$-production mode hold a large cross-section, but the fact that it produces real top-quark decays makes it a difficult background. 
The two $W$-bosons, resulting from the $t$-quark and $\overline{t}$-quark decay, again can decay leptonically or hadronically. 
If both $W$-bosons decay leptonically, the dileptonic final state, can mimic the signal signature if one of the leptons stays undetected. 
For the semileptonic decay, where one $W$-boson decays leptonically and one hadronically, the final state signature can emulate the signal if only two $b$ jets get detected.
The fully hadronic decay is very unlikely to be mistaken for the signal. 
Likewise the single top-quark production mode in the t-channel and the associated $tW$ production also show real top quark decays in their signatures.
For the t-channel the final state reproduces the signal signature if a $b$ jet or misidentified lighter quark jet gets associated with the event.
In the associated $tW$-production, the $b$-quark turns into a $t$-quark by emitting a $W$-boson.
If this $W$-boson decays hadronically and produces a $b$ jet or a jet that gets misidentified, the final state signature is the same as the signal signature.
If the $W$-boson decays leptonically and two $b$ jets or misidentified jets get associated with the event, the W+jets background also mimics the signal.
In the case of multijet background, leptons from heavy flavor decays, electrons from photon conversion or jets misidentified as leptons paired with two $b$ jets or misidentified jets lead to the signal signature.



\subsection{Previous Measurements}

\begin{itemize}
    \item Beschreibe welche Unsicherheiten genau die vorherige Messung verschlechtert
\end{itemize}

In \autoref{fig:prev_measure} previous measurements of the three different single top-quark production modes at different center of mass energies $\sqrt{s}$ are shown, taken at the ATLAS and CMS experiments at the LHC with data from $pp$-collisions.

% Das Bild muss ja eh raus
\begin{figure}
    \center
    \includegraphics[width=0.85\textwidth]{imgs/fig_10.png}
    \caption{Chart of all measurements taken of the three different sinlge top production modes at the ATLAS and CMS experiments \cite{prev_measure}}
    \label{fig:prev_measure}
\end{figure}

The most recent measurement of the s-channel was done by ATLAS with a center of mass energy of
$\sqrt{s}=\SI{13}{\TeV}$, with data collected between the years 2015 and 2018, corresponding to the Run 2 dataset at an integrated luminosity of \SI{140}{\femto\barn^{-1}}\,\cite{lastmeasure_atlas}. 
The cross section of the measured s-channel resulted in $\sigma_\rm{s-chan} = 8.2 \pm 0.6 (\text{stat.})^{+3.4}_{-2.8}(\text{syst.})\unit{\pico\barn}$, in comparison to the theoretical estimate of $\sigma_\rm{s-chan}^{\rm{theo.}}=\asym{10.32}{0.40}{0.36}{\pico\barn}$\,\cite{schan_theory_xsec}. 
This is equivalent with an observed (expected) signal significance of $3.3(3.9) \sigma$ over the background only hypothesis .
The analysis strategy consisted of a preselection and then a split into one signal region and three validation regions, for better modelling of the background processes.
To extract the signal strength from the events, a profile likelihood fit was done, with a matrix element method (MEM) discriminant.
The limiting factors of the previous measurement where noticeably the systematic uncertainties, where $\ttbar$ modelling and signal modelling uncertainties, as well as detector modelling uncertainties, had the biggest impact (Irgendwie muss ich das ja noch zeigen? Wie soll ich da mit der Systematics Tabelle aus dem Paper umgehen?).