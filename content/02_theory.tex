\chapter{Properties of the Top Quark within the Standard Model}
\label{chapter2}

\section{Overview of the Standard Model}
\label{overview_sm}

The SM of particle physics is a gauge invariant theory, which describes the fundamental particles and the interactions between them. 
It was developed throughout the 20th century  (citation) by combining the discoveries of quantum mechanics and special relativity into a quantum field theory. 
Three of the four fundamental forces, electromagnetism, the strong interaction and the weak interaction are characterized by it, only excluding gravity. 
Whereas electromagnetism and gravity act on infinite ranges, the strong and weak force act on subatomic scales. 

The SM divides the fundamental particles into two main categories. 
First fermions, which constitute all known matter and carry half integer spin.  
Second bosons, which act out the interactions between fermions and carry a whole integer spin. 
Further, bosons can be divided into vector bosons, with spin 1 and a scalar boson, the Higgs boson ($H$) with spin 0. 
The fermions are comprised of quarks and leptons, where both interact with the electromagnetic (EM) force but only quarks experience the strong force. 
Neutrinos, which take part in the lepton category, only interact via the weak force. 
The fundamental particles are further split into three generations in relation to their masses and time of discovery. 
The first generation is comprised of the electron ($e$) and the corresponding electron neutrino ($\nu_\rm{e}$) for the lepton part, and the up quark ($u$) as well as the down quark ($d$). 
The second generation leptons include the muon ($\mu$) and the muon neutrino ($\nu_\rm{\mu}$), while the charm quark ($c$) and the strange quark ($s$) constitute the second generation quarks. 
The third generation includes the heaviest fermions, with the tau lepton ($\tau$) and its corresponding tau neutrino ($\nu_\rm{tau}$) on the lepton side and the top quark ($t$) and bottom quark ($b$) on the quark side. 
The electric charge defines the interaction strength with the EM force. All charged leptons posses an electric charge of -1 elementary charge (e). 
Whereas up-type quarks ($u, c, t$) carry a charge of $+\sfrac{2}{3} \, \rm{e}$ and down-type quarks ($d, s, b$) carry a charge of $-\sfrac{1}{3} \, \rm{e}$. 
For every fermion there exists an antiparticle counterpart with oppsite electric charge and spin. 
Especially the first generation of fermions constitute to the overwhelming part of matter, where protons and neutrons are build with the $u$- and $d$-quarks and electrons as part of atoms.

The force carrying bosons include the photon ($\gamma$) for the EM interaction, the $\rm{W}^{\pm}$ and $\rm{Z}$ bosons for the weak interaction and eight gluons with different colour for the strong interaction. 
The photon is mass- and chargeless, with it coupling to every particle possesing an electric charge. 
The $\rm{W}^{\pm}$  and $\rm{Z}$ bosons of the weak interaction carry relatively high mass, which is the reason for its narrow interaction range. 
The weak force acts on every fermion, where the $\rm{W}^{\pm}$ bosons couple to the weak isospin and the $\rm{Z}$ boson couples to the weak hypercharge, a combined value of the electric charge and weak isospin.   
Gluons, mediating the strong interaction, are massless as well and couple to any particle carrying a colour charge, either quarks or glouns themselves. 
There are the three base colours red, green and blue with a respective anticolour counterpart. Gluons always carry two colour charges, one base colour and one anticolour, resulting in eight different possible permutations. 
Quarks carry only one colour charge. 
Colour confinement states that every stable composite particle has to be colorless, which is the reason for the finite range of the strong interaction.
These stable composites are called hadrons and are further split into mesons and baryons. 
Mesons are colour neutral quark-anitquark pairs. Baryons are colour neutral composites of three quarks. 
The Higgs Boson, proposed by Peter Higgs in 1964 \cite{higgs}, was discovered in 2012 in a combined effort of the ATLAS \cite{atlas_higgs} and CMS \cite{cms_higgs} experiment at the LHC.
As a scalar boson it carries spin 0, no electric charge and is the heaviest boson ($m_\rm{H}=\SI{125.20 \pm 0.11}{\GeV}$ \cite{higgs_mass}). 
It mediates none of the fundamental forces, but by coupling to it fermions and the $\rm{W}^{\pm}$ and $\rm{Z}$ boson gain their mass. 


\begin{figure}
    \centering
    \includegraphics[width=0.7\textwidth]{imgs/SM1.png}
    \caption{The fundamental particles of the Standard Model, showing which forces act on them respectively \cite{UZH_SM}}
    \label{sm_table}
\end{figure}

\section{The Top Quark}

First postulated by Makoto Kobayashi and Toshihide Maskawa in 1973 and later discovered at Fermilab with the CDF and D\O \, experiments of the Tevatron collider in 1995 in $p\overline{p}$ collisions, the top quark is the last discovered quark in the SM. 
As up-type quark in the third generation, it carries the heaviest mass of all quarks at $m_\rm{t}= \SI{172.53 \pm 0.29}{\GeV}$ (citation), it posseses a large decay width of $\Gamma_{t}= \asym{1.42}{0.19}{0.15}{\GeV}$ (citation). 
A large decay width leads to a short mean life time of $\tau = \sfrac{\hbar}{\Gamma_\rm{t}} \approx \SI{5e-25}{\s}$. 
Unlike other quarks, which hadronise due to colour confinement, the top quark decays befor hadronisation is possible, due to its short lifetime. 
The tops properties, like spin information or kinetic properties, are then to be decoded by studying the decay particles, which makes its analysis especially interesting. 
The top quark decays almost exclusively into a $\rm{W}$-boson and a $b$-quark.   

\subsection{Single top-quark production in the s-channel}

The s-channel is one of the electroweak single top-production modes. 
In the single top-quark s-channel, a top-quark and a bottom anitquark are created from a virtual (off-shell) $\rm{W}$-boson, after the annihilation of a quark and antiquark. 
The s-channel shows a $tWb$ vertex, where the CKM matrix element $V_\rm{tb} \approx 1$ is involved.
Therefore the s-channel is a strong probe for the electroweak properties of the top quark. 
In the LHC, the annihilating partciles are predominantly valence up quarks ($u$) and antidown ($\overline{d}$) from the pool of sea quarks in the colliding protons. 
The parton density function (PDF) is a representation of the probability distrubtion with wich a parton carries a fraction of the momentum of a hadron.    




\begin{itemize}
    \item faktorisierungstheorem als "aufhänger" (nachlesen)
    \item s-channel is our production mode of interest
    \item tt bar via gluon gluon fusion is predominiant mode
    \item s-channel has the smallest cross section
    \item Anti quarks from sea quarks in the pdf
    \item Explain PDFs to explain sea and partons
    \item How does the theoretical cross section get calculated? Via factorisation theorem
    \item s-channel feynman diagram
    \item what does the signature look like?
\end{itemize}


\begin{figure}
    \begin{subfigure}{0.5\textwidth}
        \begin{tikzpicture}
            \begin{feynman}
              \vertex (a);
              \vertex [above left=of a] (i1) {$u$};
              \vertex [below left=of a] (i2) {$\overline{d}$};
              \vertex [right=of a] (b);
              \vertex [above right=of b] (f1) {$t$};
              \vertex [below right=of b] (f2) {$\overline{b}$};
              % \vertex [above right=5mm and 1cm of f1] (c1) {$W^+$}; 
              % \vertex [below right=5mm and 1cm of f1] (c2) {$b$};
              % \vertex [above right=5mm and 15mm of c1] (d1) {$\overline{e}, \, \overline{\mu}$};  
              % \vertex [below right=5mm and 15mm of c1] (d2) {$\nu_e , \, \nu_{\mu}$};

              \diagram* {
              (i1) -- [ fermion] (a) -- [ fermion] (i2);
              (a) -- [boson, edge label={$W^+$}](b);
              (f2) -- [fermion] (b) -- [ fermion] (f1);
              % (c1) -- [boson] (f1) -- [fermion] (c2);
              % (d1) -- [fermion] (c1) -- [fermion] (d2);
              };

            \end{feynman}
        \end{tikzpicture}
    \end{subfigure}
    % \hfill
    \begin{subfigure}{0.5\textwidth}
        \begin{tikzpicture}
            \begin{feynman}
              \vertex (a);
              \vertex [right=of a] (b);
              \vertex [above right=7mm and 15mm of b] (c1);
              \vertex [below right=of b] (c2) {$b$};
              \vertex [above right=8mm of c1] (d1) {$\overline{e}, \, \overline{\mu}$};
              \vertex [below right=8mm of c1] (d2) {$\nu_e, \, \nu_\mu$};

              \diagram* {
                (a) -- [edge label={$t$}, fermion](b);
                (c1) -- [edge label={$W^+$}, boson] (b) -- [fermion] (c2);
                (d1) -- [fermion] (c1) -- [fermion] (d2);
              };
            \end{feynman}
        \end{tikzpicture}
    \end{subfigure}
    \caption{Feynman Diagram of the singlte top-quark s-channel production mode and further decay of the top-quark.}
    \label{fig:schan_feynman}
\end{figure}


\subsection{Top quark decay}

\begin{itemize}
    \item decay via weak interaction
    \item decay into W boson and bottom quark because of CKM $V_tb$ approx 1
    \item W boson can decay hadronically 2/3 or leptonically 1/3.  Hadronic channel has bigger phase space because of more color charge possibilities
    \item dileptonic, semileptonic or hadronic challenge
    \item hadronic decay can be detected through jets
    \item leptonic decay leads to undetected neutrino. This leads to $E_t^\text{miss}$(how is it calculated?)
    \item What do the different channel signatures look like?
    \item We use semileptonic channel because of multijet background
\end{itemize}

\subsection{Background Processes}

Hier kann ich im Prinzip alles aus der Präsi rein hauen

\begin{itemize}
    \item What is background (Processes with similar signature)
    \item What are the most relevant backgrounds
    \item Feynman diagramms of background
    \item erklären wann background als signature gesehen werden kann
    \item Also im Prinzip präsi. Bei Noah steht nicht viel 
    \item Beschreibe welche Unsicherheiten genau die vorherige Messung verschlechtert
\end{itemize}

\subsection{Previous measurement}

In \autoref{fig:prev_measure} previous measurements of the three different single top-quark production modes at different center of mass energies $\sqrt{s}$ are shown, taken at the ATLAS and CMS experiments at the LHC with data from $pp$-collisions.

\begin{figure}
    \center
    \includegraphics[width=0.85\textwidth]{imgs/fig_10.png}
    \caption{Chart of all measurements taken of the three different sinlge top production modes at the ATLAS and CMS experiments \cite{prev_measure}}
    \label{fig:prev_measure}
\end{figure}

The most recent measurement of the s-channel was done by ATLAS with a center of mass energy of
$\sqrt{s}=\SI{13}{\TeV}$, with data collected between the years 2015 and 2018, corresponding to the Run 2 dataset at an integrated luminosity of \SI{140}{\femto\barn^{-1}}. 
The cross section of the measured s-channel resulted in $\sigma_\rm{s-chan} = 8.2 \pm 0.6 (\text{stat.})^{+3.4}_{-2.8}(\text{syst.})\unit{\pico\barn}$, in comparison to the theoretical estimate of $\sigma_\rm{s-chan}^{\rm{theo.}}=\asym{10.32}{0.40}{0.36}{\pico\barn}$. 
This is equivalent with an observed (expected) signal significance over the background only hypothesis of $3.3(3.9) \sigma$.
The analysis strategy consisted of a preselection and then a split into one signal region and three validation regions, for better modelling of the background processes.
To extract the signal strength from the events, a profile likelihood fit was done, with a matrix element method (MEM) discriminant.
The limiting factors of the previous measurement where noticeably the systematic uncertainties.